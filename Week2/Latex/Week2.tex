\documentclass[12pt]{report}
\usepackage[utf8]{inputenc}
\usepackage[margin=1in]{geometry}
\usepackage{parskip}
\usepackage{xcolor}
\usepackage{url}
\usepackage{framed}
\usepackage{hyperref}
\definecolor{LightGray}{gray}{0.97}

\renewcommand\thesection{Exercise \arabic{chapter}.\arabic{section}: }
\author{The Cool Kids}

%%%%%%%%%%%%%%%%%%%%%%%%%%%%%%%%%%%%%%
% Setting

\newcommand{\weeknr}{2}
\newcommand{\weektitle}{Python}

%%%%%%%%%%%%%%%%%%%%%%%%%%%%%%%%%%%%%%

\title{Big Tools for Complicated Data \\Week \weeknr: \weektitle}
\usepackage{listings}

% Overall settings
\lstset{
  basicstyle=\ttfamily\small,
  showspaces=false,
  showstringspaces=false,
  showtabs=false,
  breaklines = true,
  tabsize=4
} 
  
% Custom colors
\definecolor{deepblue}{rgb}{0,0,0.5}
\definecolor{deepred}{HTML}{94558D}
\definecolor{deepgreen}{HTML}{008080}
\definecolor{commentgrey}{HTML}{808080}


% Python style for highlighting
\lstdefinestyle{Python}{
  language=Python,
  %
  otherkeywords={self},             % Add keywords here
  % 
  % Standard highlighting
  keywordstyle=\color{deepblue},
  commentstyle=\color{commentgrey},
  stringstyle=\bfseries\color{deepgreen},
  % 
  % Special emphasis
  emph=[1]{self, __init__, __doc__, __name__, __dict__, __new__, __del__, __repr__, __str__, __bytes__, __format__,
  __lt__, __le__, __eq__, __ne__, __gt__, __ge__, __hash__, __bool__, __getattr__, __getattribute__,
  __setattr__, __delattr__, __dir__, __get__, __set__, __delete__},
  emphstyle=[1]{\color{deepred}},
  emph=[2]{def, print, class, True, False},
  emphstyle=[2]{\bfseries\color{deepblue}},
  %
  showstringspaces=false            % 
}

\begin{document}

\maketitle

\section*{Info}

All code is available at \url{https://github.com/North-Guard/BigToolsComplicatedData/tree/master/Week2}. 
The directory includes 5 scripts, where some of the scripts handle the same problem but are optimized and both kept for later reference. \\

Files:
\begin{itemize}
\item \lstinline|exercise_1.py|
\item \lstinline|exercise_2.py|
\item \lstinline|exercise_2_optimized.py|
\item \lstinline|exercise_3.py|
\item \lstinline|exercise_3_optimized.py|
\end{itemize}

\setcounter{chapter}{1}

\clearpage
\section{lists and recursion}
Write a script that takes an integer N, and outputs all bit-strings of length N as lists. For example: $3 -> [0,0,0], [0,0,1],[0,1,0],[0,1,1],[1,0,0],[1,0,1],[1,1,0],[1,1,1]$. As a sanity check, remember that there are $2^N$ such lists.

Do not use the bin-function in Python. Do not use strings at all. Try to solve this using only lists, integers, if-statements, loops, functions and the map function (you don’t need all these things).

\textbf{Solution:}
Line 5 computes the sequences on the subproblems for smaller $N$. The method then loops through these lines and creates the extensions by extending the sequences with $0$'s and $1$'s.
\begin{framed}
\begin{lstlisting}[style=Python, numbers=left]
def bits(N):
    if N == 1:
        return [[0], [1]]
    else:
        l = bits(N-1)
        l_all = []
        for i in range(len(l)):
            l0 = []
            l1 = []
            l0.extend(l[i])  # extend is used to avoid making a list of lists, but instead appending the elements of the list to the new list
            l0.extend([0])
            l1.extend(l[i])
            l1.extend([1])
            l_all.append(l0)
            l_all.append(l1)

        return l_all

N = 3
s = bits(N)

print(s)
print(len(s))
\end{lstlisting}
\end{framed}

\clearpage
\section{bag of words}
Write a script that takes this file (from this Kaggle competition), extracts the \texttt{request\_text} field from each dictionary in the list, and construct a bag of words representation of the string (string to count-list).

There should be one row pr. text. The matrix should be $N x M$ where $N$ is the number of texts and $M$ is the number of distinct words in all the texts.  The result should be a list of lists ($[[0,1,0],[1,0,0]]$ is a matrix with two rows and three columns).

\textbf{Solution:}
The following scripts performs the task with comments on segments. 
\begin{framed}
\begin{lstlisting}[style=Python, numbers=left]
from time import time
import matplotlib.pyplot as plt
import numpy as np
plt.close("all")

# Load the data
with open('Week2/pizza-train.json', 'r') as file:
    lines = [line.split() for line in file.readlines()]

# Get request_text's
lines = [line[1:] for line in lines 
	     if line[0] == '"request_text":']
n_lines = len(lines)

# Find all distinct words in vocabulary
words = set()
idx = 0
for line in lines:
    text = line
    words.update(text)

# Make a word-list and mapping to indices
words = sorted(list(words))
word2idx = {word: idx for idx, word in enumerate(words)}
n_words = len(words)

# Bag-of-words list of lists
bow = [[0] * n_words for _ in range(n_lines)]
for line_nr, line in enumerate(lines):
    text = line
    bow_row = bow[line_nr]
    for word in text:
        ix = word2idx[word]
        bow_row[ix] = 1
\end{lstlisting}
\end{framed}

\clearpage
\section{Make Your Own}
Make your own exercise to teach people how to use python dictionaries. Provide a solution for the exercise. 

\textit{Use dictionaries to speed up the recursive function from exercise 1, e.g. when using recursive functions with a large number of repetitive recalculations, e.g. a large $N$ in this case. Specifically, build a dictionary called bits\_dict into the function, so that all the lists for values <= the biggest N that has been calculated can be obtained from the dictionary by calling bits\_dict[value $<=$ the biggest calculated so far]. Last but not least, think about why a dictionary can speed up recursive functions.}

\textbf{Solution:}
The code-snippet below shows a solution to the exercise. Line 3 defines the dictionary and initializes it with the base case where $N=1$. When given a number, \texttt{bits} first check whether it already knows the solution to the problem and returns this problem if so. Otherwise it creates a list by extracting all lists from the subproblem for $N-1$ and combining these with $0$ and $1$ (lines 7-11). 

\begin{framed}
\begin{lstlisting}[style=Python, numbers=left]
import time

bits_dict = {1: [[0], [1]]}

def bits(n):
    if n not in bits_dict:
        l_all = [
            [val, *li]
            for val in [0, 1]
            for li in bits(n - 1)
        ]

        # Store subproblem result
        bits_dict[n] = l_all

    return bits_dict[n]

N = 22
start = time()
s = bits(N)
end = time()

print("Sequences created : {}".format(len(s)))
print("Sequences expected: {}".format(2**N))
print("Time for creation: {:.2f}s".format(end-start))

\end{lstlisting}
\end{framed}

\end{document}